%!Tex Program = xelatex
\documentclass[12pt,twoside]{article}
\usepackage{fontspec, xunicode, xltxtra}
\usepackage[slantfont,boldfont]{xeCJK} % 允许斜体和粗体
\usepackage{geometry}
\usepackage{indentfirst}
\usepackage{lastpage}
\usepackage{multirow}
\usepackage{multicol}
\usepackage{titlesec}
\usepackage{enumerate}
\usepackage{booktabs}
\usepackage{extarrows}
\usepackage{fancyhdr,fancybox}
\usepackage[table]{xcolor}
\usepackage{float}
\usepackage{tikz}
%\usepackage{amssymb}
%\usepackage{amsmath}
%\usepackage{caption}
%\usepackage{dirtree}
%\usepackage{algorithm}
%\usepackage{algpseudocode}
\usepackage{listings}
\usepackage[colorlinks,linkcolor=black]{hyperref}

\geometry{left=2.5cm,right=2.5cm,top=2.5cm,bottom=2.5cm}

%cmd “fc-list :lang=zh-cn”
\defaultfontfeatures{Mapping=tex-text}
\setmainfont{Book Antiqua}   % 英文衬线字体 Times New Roman, Palatino Linotype
\setmonofont{Consolas}   % 英文等宽字体 Monaco
\setsansfont{Consolas} % 英文无衬线字体 Arial, Futura, Optima
\setCJKmainfont{SimSun}   % 设置缺省中文字体
\setCJKmonofont{FangSong}   % 设置等宽字体
\setCJKsansfont{SimHei}   % 设置无衬线字体 Microsoft YaHei

\newfontfamily\timesroman{Times New Roman}
\newfontfamily\consolas{Consolas}

\newfontfamily\ensong{SimSun}
\newfontfamily\enhei{SimHei}
\newfontfamily\enyahei{Microsoft YaHei}
\newfontfamily\palatino{Palatino Linotype}
\newfontfamily\nimbus{Nimbus Sans L}
\newfontfamily\enkai{STKaiti}
\newfontfamily\enfangsong{FangSong}
\newfontfamily\akaDora[Path=../]{akaDora.ttf}

\setCJKfamilyfont{song}{SimSun}
\newcommand{\song}{\CJKfamily{song}\ensong}
\setCJKfamilyfont{heiti}{SimHei}
\newcommand{\heiti}{\CJKfamily{heiti}\enhei}
\setCJKfamilyfont{yahei}{Microsoft YaHei}
\newcommand{\yahei}{\CJKfamily{yahei}\enyahei}
\setCJKfamilyfont{kaiti}{STKaiti}
\newcommand{\kaiti}{\CJKfamily{kaiti}\enkai}
\setCJKfamilyfont{fangsong}{FangSong}
\newcommand{\fangsong}{\CJKfamily{fangsong}\enfangsong}

\XeTeXlinebreaklocale "zh"
\XeTeXlinebreakskip = 0pt plus 1pt minus 0.1pt

\titlespacing*{\chapter} {0pt}{50pt}{40pt}
\titlespacing*{\section} {0pt}{3.5ex plus 1ex minus .2ex}{2.3ex plus .2ex}
\titlespacing*{\subsection} {0pt}{3.25ex plus 1ex minus .2ex}{1.5ex plus .2ex}
\titlespacing*{\subsubsection}{0pt}{3.25ex plus 1ex minus .2ex}{1.5ex plus .2ex}
\titlespacing*{\paragraph} {0pt}{3.25ex plus 1ex minus .2ex}{1em}
\titlespacing*{\subparagraph} {\parindent}{3.25ex plus 1ex minus .2ex}{1em}

\titleformat{\section}{\Large \bf \nimbus}{\thesection}{1em}{}
\titleformat{\subsection}{\large \bf \nimbus}{\thesubsection}{0.5em}{}

\setlength{\parskip}{0.5\baselineskip}
\setlength{\abovedisplayskip}{1pt}
\setlength{\belowdisplayskip}{1pt}

\usetikzlibrary{arrows,positioning}
\newcommand\nbvspace[1][3]{\vspace*{\stretch{#1}}}
\newcommand\nbstretchyspace{\spaceskip0.5em plus 0.25em minus 0.25em}
\newcommand{\nbtitlestretch}{\spaceskip0.6em}

\pagestyle{fancy}


\lstset{
    basicstyle=\small,
    backgroundcolor=\color{white},
    %keywordstyle=\color{keywordcolor}\bfseries, %\underbar,
    keywordstyle=\color{blue}\bfseries,
    %morekeywords={*,keyword_a,keyword_b},
    sensitive=true,
    identifierstyle=\small,
    showspaces=false,
    showstringspaces=false,
    showtabs=false,
    tabsize=4,
    frame=single,
    commentstyle=\color{olive} \textit,
    stringstyle=\ttfamily,
    showstringspaces=false,
    captionpos=b,
    breaklines=true,
    breakatwhitespace=true,
  }

\lstdefinelanguage{Makefile} {
  numberblanklines=false,
  keywordstyle=\color{blue}\bfseries,
  morekeywords={gedit,ls,rm,source,sudo,tar},
}

\lstdefinelanguage{LinuxTerminal} {
  numberblanklines=false,
  keywordstyle=\color{blue}\bfseries,
  morekeywords={cat,cd,chmod,gedit,ln,ls,
    make,mkdir,rm,source,sudo,tar,touch},
  morecomment=[l]{\#},
}

\lstdefinelanguage{PlainText} {
  numberblanklines=false,
}



\usepackage{tikz}
\usetikzlibrary{mindmap,trees}
\usetikzlibrary{arrows,positioning}

\begin{document}

\thispagestyle{empty}
  \begin{center}
    \bfseries
    \nbvspace[2]
    \begin{figure}[H]
      \centering
      \includegraphics[width=0.6\textwidth]{../logo.pdf}
    \end{figure}
    {\Huge GDPLS Movie} \\[10pt]
    {\LARGE\akaDora Grand Duke of Programming Language Script}\\[10pt]
    {\Huge 2017 June} \\
    \nbvspace[1]
    \Huge Meeting Record\\
    \nbvspace[1]
    \normalsize Generated \& Compiled By \XeLaTeX
    \nbvspace[3]
  \end{center}
  \newpage

  \lhead{\emph{VISION TABLE}}
  \begin{table}[H]
    \centering
    \renewcommand\arraystretch{1.3}
    \rowcolors{2}{blue!50}{blue!20}
    \begin{tabular}{lllp{25em}}
      %\rowcolors{1}{blue!80}{blue!10}
      \multicolumn{4}{c}{\heiti 版本日志}\\
      版本号 & 修改人 & 修改时间 & \multicolumn{1}{c}{说明} \\
      0.0.0 & 郑斯达 & 2017.3.23 & 会议记录\\
      1.0.0 & 方铭 & 2017.3.23 & 加封面排版\\
      &&&\\
      &&&\\ % span text width
    \end{tabular}
  \end{table}
  \newpage


\begin{flushleft}
\large
  {\bf 会议时间}: 2017年03月23日 19:00-20:28

\par {\bf 会议地点}: 至善学生活动中心公共讨论区
\par {\bf 主 持 人}:徐家豪
\par {\bf 参会人员}:陈锐煌、卢卓君、徐广晖、方铭
\par {\bf 记 录 人}:郑斯达
\par {\bf 会议议题}:构建新一代电影购票服务网站
\par {\bf 会议内容}:
2017年03月23日,徐家豪项目经理,召集客户经理卢卓君,总工程师徐广晖、工程师、配置经理郑斯达,质量保证经理方铭,技术经理徐广晖,测试工程师陈锐
煌,在至善学生活动中心公共讨论区开会,启动了构建新一代电影购票服务网站项目,会议形成如下意见:
\end{flushleft}

\section{购票体验}
流程:\\
选择影院->选择电影(查看电影的相关信息)->选择座位->买票->取票\\
(或选择电影->选择影院(查看影院的相关信息)->选择座位->买票->取票)
\par
电影的相关信息:电影的简要信息、现有的评价、电影的场次、其他信息。
\par
影院的相关信息:影院的地点距离、影院的大小场次、影院的评价、影院附近的景点美食
\par
结论:必须加入电影的相关信息与影院的相关信息。加入评论功能,只有看过电影之后才能评论。
\section{核心亮点}
\begin{enumerate}
  \item 一键购票
  \item 选择好电影和时间段
  \item 一键选择影院,场次,座位
\end{enumerate}

\section{辅助亮点}

\begin{enumerate}
  \item 好友功能:能够对自己的好友发起看电影请求
  \item 约票功能:能够与附近同样像看同一部电影的人匹配,帮助选定座位和场次。
  \item 推荐导演的其他电影
\end{enumerate}


\href{http://blog.csdn.net/myzczx/article/details/52497374#_Toc450478455}{参考链接}

\section{思维导图}

\par 电影订票系统

\begin{center}
  \begin{tikzpicture}
  \path[mindmap,concept color=black,text=white]
    node[concept] {电影订票系统}
    [clockwise from=0]
    child[concept color=green!50!black] {
      node[concept] {用户需求}
      [clockwise from=90]
      child { node[concept] {简化订票流程} }
      child { node[concept] {推荐电影} }
      child { node[concept] {综合排序比较} }
      child { node[concept] {影评互动} }
    }
    child[concept color=blue] {
      node[concept] {核心业务}
      [clockwise from=-30]
      child { node[concept] {一键购票} }
      child { node[concept] {选择好电影和时间段}}
      child { node[concept] {一键选择影院,场次,座位} }
    }
    child[concept color=orange] {
      node[concept] {辅助业务}
      [clockwise from=-70]
      child { node[concept] {好友功能} }
      child { node[concept] {约票功能}}
      child { node[concept] {一键选择影院,场次,座位} }
    }
    child[concept color=red] { node[concept] {实现}
    [clockwise from=100]
        child { node[concept] {Android}}
        child { node[concept] {HTML5}}
    };
\end{tikzpicture}
\end{center}


\end{document} 